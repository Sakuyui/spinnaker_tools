% Copyright (c) 2009-2023 The University of Manchester
%
% Licensed under the Apache License, Version 2.0 (the "License");
% you may not use this file except in compliance with the License.
% You may obtain a copy of the License at
%
%     http://www.apache.org/licenses/LICENSE-2.0
%
% Unless required by applicable law or agreed to in writing, software
% distributed under the License is distributed on an "AS IS" BASIS,
% WITHOUT WARRANTIES OR CONDITIONS OF ANY KIND, either express or implied.
% See the License for the specific language governing permissions and
% limitations under the License.

\def\FullTitle{\textsl{ybug} - System Control Tool for SpiNNaker}
\def\ShortTitle{ybug 2.0.0}
\def\Date{8 Mar 2016}
\def\Version{2.0.0}
\def\Author{Steve Temple}
\def\Email{steven.temple@manchester.ac.uk}

%
% Simple LaTeX header to generate SpiNNaker documentation
%
% Generates AppNotes if variable NoteNum is defined otherwise
% just generates a simple note. The variables TitleA and TitleB
% may be overridden (second and third lines of title heading)
% TitleC is optional and only used if defined.
%

% Copyright (c) 2009-2019 The University of Manchester
%
% This program is free software: you can redistribute it and/or modify
% it under the terms of the GNU General Public License as published by
% the Free Software Foundation, either version 3 of the License, or
% (at your option) any later version.
%
% This program is distributed in the hope that it will be useful,
% but WITHOUT ANY WARRANTY; without even the implied warranty of
% MERCHANTABILITY or FITNESS FOR A PARTICULAR PURPOSE.  See the
% GNU General Public License for more details.
%
% You should have received a copy of the GNU General Public License
% along with this program.  If not, see <http://www.gnu.org/licenses/>.

\documentclass[11pt,a4paper]{article}

\usepackage{fancyhdr}
\usepackage{color}
\usepackage{ifthen}
\usepackage{xspace}
\usepackage{listings}
\usepackage{enumitem}
\usepackage{alltt}
\renewcommand{\ttdefault}{txtt}
%\usepackage[compact]{titlesec}

\usepackage[pdftex,a4paper]{geometry}
\usepackage[pdftex]{graphicx}

\geometry{left=25mm,right=25mm,top=25mm,bottom=20mm}

\ifdefined\TitleA
\else
  \def\TitleA{SpiNNaker Group, School of Computer Science,
    University of Manchester}
\fi

\ifdefined\TitleB
\else
 \def\TitleB{\Author\ - \Date\ - Version \Version}
\fi

\ifdefined\NoteNum
 \def\HeadLeft{SpiNNaker AppNote \NoteNum}
 \def\HeadCentre{\ShortTitle}
 \def\HeadRight{Page \thepage}
 \def\Title{AppNote \NoteNum\ - \FullTitle}
\else
 \def\HeadLeft{\ShortTitle}
 \def\HeadCentre{}
 \def\HeadRight{Page \thepage}
 \def\Title{\FullTitle}
\fi

\addtolength{\headheight}{2pt}

\pagestyle{fancy}

\renewcommand{\headrulewidth}{0.6pt}

\lhead{\itshape\HeadLeft}
\chead{\itshape\HeadCentre}
\rhead{\itshape\HeadRight}

\fancyfoot{}

\newcommand{\image}[5][]
{
  \begin{figure}[#2]
  \begin{center}
  \includegraphics[#1]{#3}
  \ifthenelse{\equal{#5}{}}{}{\caption{#5}}
  \label{fig:#4}
  \end{center}
  \end{figure}
}

% A substitute for changepage/chngpage's adjustwidth: provides a changemargin
% environment.
\def\changemargin#1#2{\list{}{\rightmargin#2\leftmargin#1}\item[]}
\let\endchangemargin=\endlist 

\newenvironment{shell}
{
  \begin{changemargin}{2em}{0em}
  \begin{alltt}
  \small
}
{
  \end{alltt}
  \end{changemargin}
}

\setcounter{secnumdepth}{-1} 
\setlength{\parindent}{0pt}
\setlength{\parskip}{6pt}

\setitemize{itemsep=0pt}
\setitemize{parsep=0pt}
\setitemize{leftmargin=12pt}

\lstset{xleftmargin=2em,columns=flexible,basicstyle=\ttfamily\small}

\begin{document}

\begin{center}
\setlength{\parskip}{0em}
{\Large\bfseries\Title}
\par
\vspace{5mm}
{\large\itshape\TitleA}
\par
\vspace{2mm}
{\large\itshape\TitleB}
\ifdefined\TitleC
\par
\vspace{2mm}
{\large\itshape\TitleC}
\fi
\end{center}


\newcommand{\ybug}{\textbf{\textsl{ybug}}}

\section{Introduction}

\ybug\ is a program which runs on a host computer (workstation) and
provides an interactive text-based interface to a SpiNNaker system.
It allows the system to be bootstrapped and programs and data to be
downloaded. There are a number of low-level debugging features such as
the ability to inspect and change memory in any SpiNNaker chip in the
system. \ybug\ communicates with the target system using a protocol
based on UDP/IP and so can control systems located anywhere on the
Internet.

\subsection{Installation and dependencies}

\ybug\ is written in Perl and uses a number of locally developed Perl
libraries. Installation just requires the copying of the main source
file and the libraries to a suitable place on your workstation and the
setting of some environment variables to reference these. \ybug\ needs
the library \texttt{String::CRC32} which is commonly installed but
can be also found at CPAN or as package \texttt{libstring-crc32-perl}
(Ubuntu, etc) or \texttt{perl-String-CRC32} (Fedora, etc).

There is also one optional, but very useful, dependency on the Perl
library which interfaces to the GNU ReadLine library. This provides
command history and filename completion and requires the installation
of the library \texttt{Term::ReadLine::Gnu}. This can be downloaded
either from CPAN or via a package such
as \texttt{libterm-readline-gnu-perl} (Ubuntu, etc)
or \texttt{perl-Term-ReadLine-Gnu} (Fedora, etc).

\subsection{Background}

\ybug\ began as a simple hack to talk to early SpiNNaker systems. Like
many hacks, it has not (yet) been superseded. This note documents
version \Version\ of \ybug.

\section{Starting \ybug\ and the Command Line Interface (CLI)}

\ybug\ is started from the Unix shell by giving the command \texttt{ybug}
with a single compulsory argument which is the IP address of the
SpiNNaker system that you wish to control. The IP address can be
numeric or a host name which will be looked up in the usual way. \ybug\
accepts options which begin with \texttt{-}. The option \texttt{-help}
will list all options. If you don't have the ReadLine library loaded
and \ybug\ complains about this, the \texttt{-norl} flag may help.

When the program starts you are presented with a prompt which has four
components. The first is the hostname or IP address which was given
when the program was started. The second, third and fourth are three
numbers which indicate which chip and core on the SpiNNaker system
certain \ybug\ commands will be directed to.

At this point you can type commands which will be executed by
the \ybug\ CLI. A line beginning with the character \texttt{\#} will
be ignored and can be used for comments. Otherwise the first non-blank
field on the line is interpreted as a command and further fields are
interpreted as arguments to the command. If the ReadLine library is in
use then completion of the command name (the first item on the command
line) is attempted as is filename completion on subsequent items on the
line.

Some commands not directly related to SpiNNaker are as follows

\begin{itemize}

\item
\texttt{?} - provides a short-form list of all commands which are available.

\item
\texttt{help [<name>]} - provides a long list of all commands, their
arguments and purpose. If a valid command name is given as an argument,
just that command is listed.

\item
\texttt{<command> ?} - performs the same function as \texttt{help <command>}

\item
\texttt{version} - displays the \ybug\ version number

\item
\texttt{expert} - enables some commands which are rarely required or
may be dangerous if used carelessly

\item
\texttt{echo <string>} - echoes the string to the console. There
is no newline printed at the end of the string but the characters
\texttt{$\backslash$n} will generate a newline anywhere in the string.

\item
\texttt{pause <string>} - echoes the string to the console (like
\texttt{echo}) and then waits for the user to press \texttt{Enter}.

\item
\texttt{@ <file.F> [quiet]} - reads subsequent commands from the
supplied file. Commands are echoed unless the \texttt{quiet} flag is
provided. This command can be nested up to 10 deep.

\item
\texttt{quit} - exits \ybug\

\end{itemize}

\subsection{Arguments}

\ybug\ commands take a variety of arguments. These may be file names,
numbers in decimal or hex, IP addresses, etc. The format of each
argument required by a command is documented in the help text where
this is possible. In the help text, argument names are followed by a
character which indicates what form the argument should take.

Where a hexadecimal number is expected, a preceding ``0x'' is not
necessary and is ignored if provided. Where a decimal number is
expected, a preceding ``0x'' is allowed and the number converted.

\begin{tabular}{p{0.5cm} p{0.5cm} l}
& D & Decimal number \\
& X & Hexadecimal number \\
& R & Real number \\
& F & File name \\
& M & MAC address \\
& P & IP address \\
\end{tabular}

\subsection{Selecting a core or chip}

Most \ybug\ commands cause communication to take place with the
SpiNNaker system. Some commands need to communicate with a particular
core on a particular chip. Other commands need to communicate with the
monitor processor on a particular chip while others need to
communicate with the monitor processor on the \textbf{root chip} which
is usually the one connected to the system's Ethernet interface.

SpiNNaker chips are addressed with a pair of coordinates which are
their X and Y position in the grid of chips. The chip at coordinate
(0, 0) is normally the root chip. The range of X and Y coordinates
will vary according to the size of the SpiNNaker system.  SpiNNaker cores
are addressed as a number in the range 0 to 17.  Core 0 is the
monitor processor and cores 1 to 17 are application processors on
which application programs are run. The current X, Y and Core settings
are shown in \ybug's prompt.

The \ybug\ command \texttt{sp} is used to select the chip and/or core.
It can have up to three arguments as follows

\begin{tabular}{p{0.5cm} p{5cm} l}
& \texttt{sp root} & ChipX = 0, ChipY = 0, Core = 0 \\
& \texttt{sp <core>} & Core = core \\
& \texttt{sp <X> <Y>} & ChipX = X, ChipY = Y, Core = 0 \\
& \texttt{sp <X> <Y> <core>} & ChipX = X, ChipY = Y, Core = core \\
\end{tabular}

\ybug\ commands which know that they need to talk to a specific chip or
core which is in conflict with the currently selected chip/core will
ignore the relevant parts of the selection.

\section{Bootstrapping a SpiNNaker system}

After a SpiNNaker system has been powered-on or reset, it needs to be
bootstrapped by loading its control software (a program
called \textbf{SC\&MP}). Once the system has been bootstrapped it
should not normally be necessary to reboot it unless an application
program goes badly wrong and corrupts some critical data on a chip. At
this point the system will normally need to be reset and rebooted.

The \ybug\ command to reboot a system is \texttt{boot} and it needs two
arguments. The first is the name of a boot file and the second is the
name of a configuration file which is used to configure the bootstrap
for the particular system which is being booted. A set of standard
configuration files will be supplied with your system. For example, if
you have a SpiNN-3 board you should use the configuration file
\texttt{spin3.conf}. Various operating parameters of the system, such
as the clock speed of the processors, can be configured by editing
the config file. A typical boot command looks like this.

\begin{tabular}{p{1cm} p{12cm}}
& \texttt{> boot scamp.boot spin4.conf}
\end{tabular}

The boot and config files are searched for using the path in the
environment variable \texttt{\$SPINN\_PATH}. This variable must be set
(and exported) otherwise the files will not be found (ie there is no
default search location).

Booting may take a couple of seconds and you should see a message
confirming that the bootstrap was successful. If not, consider
resetting the system before trying again.

\section{Commands to control \ybug}

These commands control the way that \ybug\ operates and do not directly
interact with the SpiNNaker system.

\begin{itemize}

\item
\texttt{debug <num.D>} - sets a debug variable which controls how much debug
information is displayed as \ybug\ operates. 0 means no debug and 1
through 4 provide increasing amounts of information. The information
is mostly related to data packets going to and from SpiNNaker. Without
an argument, the current setting is displayed.

\item
\texttt{sleep <time.R>} - causes \ybug\ to pause for the
specified number of seconds. Can be useful in command files where time
needs to be allowed between successive commands. If no time is given,
1 second is used.

\item
\texttt{timeout <time.R>} - sets the timeout (in seconds) for responses from
SpiNNaker when commands are sent to it. The default is 0.5 seconds and
this is likely to be adequate for most connections.  Without an
argument, the current setting is displayed.

\end{itemize}

\section{Commands to load and control Applications}

A SpiNNaker application is a program which runs on a single SpiNNaker
core. Many cores may be loaded with the same application. While an
application is running on SpiNNaker it has an Application ID which is
allocated when the application is loaded. The AppID is a number in the
range 16 to 254. AppIDs below 16 and above 254 are reserved for system
use. The AppID is used to control the application as a whole while it
runs on SpiNNaker and also serves to identify shared resources on the
chip which are currently being used by the application.

A SpiNNaker application is stored in an APLX file and files of this
type are loaded into SpiNNaker to start an application running.

\begin{itemize}

\item
\texttt{app\_load <file.F> <region> <cores> <app\_id.D> [wait]} -
loads an application onto the specified set of cores on the chips in
the specified region. The application is contained in the file which
should be in APLX format.

The region specifies a set of chips ranging from all chips to a single
chip. The definition of regions on SpiNNaker is documented elsewhere
but some examples are given here.

\begin{tabular}{p{0.5cm} p{2cm} p{12cm}}
& \texttt{all} & All chips \\
& \texttt{@X,Y} & Chip (X, Y) \\
& \texttt{.} & Current chip (as selected by \texttt{sp}) \\
& \texttt{0.0.0} & 16 chips bounded by (0,0), (0,3), (3,3), (3,0) \\
\end{tabular}

The cores can be specified as a comma separated list or a minus
separated range, a combination of these two or the
string \texttt{all}. Some examples

\begin{tabular}{p{0.5cm} p{2cm} p{12cm}}
& \texttt{1} & core 1 \\
& \texttt{1-4} & cores 1 to 4 \\
& \texttt{7,9} & cores 7 and 9 \\
& \texttt{1-4,7,9} & cores 1 to 4 and 7 and 9 \\
& \texttt{all} & all cores (1 to 17)\\
\end{tabular}

The application starts to load onto all cores on a chip at the same
time but variations in access to the shared memory where the APLX file
is loaded may mean that not all cores start to run the application
simultaneously. Similarly, each chip will start to load at a different
time. Chips further away from the root chip start to load later. If
this is an issue, there are various software techniques available to
synchronise start-up.

The application is given the supplied AppID when it starts. You should
not load another application with the same AppID into the system while
the original one is still running. Applications remain in the system
even after they terminate (ie exit \texttt{c\_main}). This allows
debugging and inspection of their state but also means that they must
be explicitly removed.

If the \texttt{wait} flag is specified, the application is loaded
but does not start to execute until it receives a \texttt{start}
signal. This is useful for loading many different applications
and having them all start at the same time.

\item
\texttt{app\_stop <app\_id.D>[-<app\_id.D>]} - causes the application
with the given AppID to stop on all chips on which it is running. It
will be replaced with a default system application and all shared
resources that it claimed during execution will be freed. A range of
AppIDs may be given subject to certain restrictions. The number of
AppIDs in the range must be a power of 2 and the lower AppID must
be a multiple of that power of 2. Examples of valid ranges are
\texttt{16-23}, \texttt{64-95}, \texttt{80-81}.

\item
\texttt{app\_sig <region> <app\_id.D>[-<app\_id.D>] <signal>
[<state>]} - sends a signal to all cores running AppIDs in the given
range in chips in the given region. This feature is still in
development but it allows signals similar to Unix signals to be sent
to running applications and also allows counting how many cores in the
application are in a particular execution state. For example to count
cores running AppId 16 which are in state \texttt{RUN} and to count
cores running AppID 0 which are in state \texttt{IDLE} (the default
state)

\begin{tabular}{p{1cm} p{10cm}}
& \texttt{> app\_sig all 16 count run} \\
& \texttt{> app\_sig all 0 count idle} \\
\end{tabular}

Some valid signals are

\begin{tabular}{p{0.5cm} p{2cm} p{12cm}}
& \texttt{stop} & Terminate application and clean up \\
& \texttt{start} & Start an application when \texttt{wait} was given in \texttt{app\_load} \\
& \texttt{sync0} & Proceed past barrier (eg on API start-up) \\
& \texttt{count} & Count cores in a given state \\
\end{tabular}

\item
\texttt{ps <core.D>|x|d} - displays the status of all cores on a chip or
a single core. When used without arguments, the \texttt{ps} command
displays the state of every core on the chip showing the application
it is running, its state, how long it has been running and when it was
loaded.

With the argument \texttt{d} or \texttt{x} it displays the four user
variables associated with each core in decimal or hex. This is a
useful way of getting debugging and status information out while an
application is running.

With a numeric argument (a core number) the command displays a more
detailed set of information about a single core. Where the core has
crashed this will include a register dump which may help with
diagnosing the problem.

\end{itemize}

\section{Commands to load, dump, inspect and alter memory}

These commands allow the memory of many chips or a particular chip or
core to be displayed and altered. Before those commands which relate
to a specific chip or core are issued, the appropriate chip and core
must be selected (with the \texttt{sp} command). Some memory on a chip
(eg SDRAM) can be accessed from any core whereas other memory (eg
DTCM) can only be accessed if the appropriate core is selected.

Note that memory-mapped peripherals may also be accessed with these
commands but great care must be taken to only access them in valid
ways (ie using the correct data size and alignment) otherwise a system
crash may result.

\begin{itemize}

\item
\texttt{smemw <addr.X>} - displays 256 bytes of memory as a hex dump.
The memory is loaded as 64 words starting at the supplied address and
the display is organised in word format.

\item
\texttt{smemh <addr.X>} - displays 256 bytes of memory as a hex dump.
The memory is loaded as 128 half-words starting at the supplied
address and the display is organised in half-word format.

\item
\texttt{smemb <addr.X>} - displays 256 bytes of memory as a hex dump.
The memory is loaded as 256 bytes starting at the supplied address and
the display is organised in byte format. Where the byte has a
printable ASCII representation, it is also shown as a character.

\item
\texttt{sw <addr.X> [<data.X>]} - displays (one argument) or sets (two
arguments) a single word in memory at the given address (which should
be word aligned).

\item
\texttt{sh <addr.X> [<data.X>]} - displays (one argument) or sets (two
arguments) a single half-word in memory at the given address (which
should be half-word aligned).

\item
\texttt{sb <addr.X> [<data.X>]} - displays (one argument) or sets (two
arguments) a single byte in memory.

\item
\texttt{sload <file.F> <addr.X>} - reads a file and copies its
contents into memory beginning at the specified address. The data is
written as bytes so that any length of file may be used.

\item
\texttt{sdump <file.F> <addr.X> <len.X>} - reads the specified length of
SpiNNaker memory starting at the given address and copies it to a
file. The data is read as bytes. Note that the length is specified in
hexadecimal.

\item
\texttt{sfill <from.X> <to.X> <word.X>} - fills SpiNNaker memory with
the specified word, beginning at the specified \texttt{from} address
and ending at the word below the \texttt{to} address. Both addresses
should be word aligned.

\item
\texttt{data\_load <file.F> <region> <addr.X>} - writes the content of
the specified file to memory at the given address in all chips
specified by \texttt{region}. This is useful if the same data has to
be sent to many chips simultaneously and the data is going to a shared
area of memory (ie not ITCM or DTCM). Writing to a single chip is more
efficient using \texttt{sload}.

\end{itemize}

The next three commands are only available in \texttt{expert} mode.

\begin{itemize}

\item
\texttt{gw <addr.X> <data.X>} - writes the specified data as a word
to the given address on every SpiNNaker chip in the system. Note that
only a limited set of addresses is supported, to allow access to some
important peripherals and parts of memory. Probably not for everyday
use but can be useful for debugging.

\item
\texttt{gh <addr.X> <data.X>} - writes the specified data as a half-word
to the given address on every SpiNNaker chip in the system. See \texttt{gw}
for further details.

\item
\texttt{gb <addr.X> <data.X>} - writes the specified data as a byte
to the given address on every SpiNNaker chip in the system. See \texttt{gw}
for further details.


\end{itemize}

\section{Commands to control IPTags}

An IPTag is a numeric identifier which is used to index a table in
a chip with an Ethernet interface. Entries in the table contain an
IP address and port number. They are used to direct SDP packets which
arrive at the Ethernet-connected chip from within SpiNNaker bearing an
IPTag to the appropriate IP address and port using the UDP protocol.

A typical use is to direct debugging output from \texttt{io\_printf}
functions executed on application cores to a host system which then
displays the debug text. IPTag 0 is normally used for this purpose.
The \texttt{iptag} command is used to control the IPTag tables.

\begin{itemize}

\item
\texttt{iptag} - without arguments, the \texttt{iptag} command displays
the contents of the IPTag table. Only valid entries are displayed. The
table has two sections and the size of these sections is shown by this
command as well as the timeout which is applied to transient IPTags.
Lower entries in the table are permanent tags which are set up
explicitly either by a host or by a SpiNNaker application. Higher
entries are transient and only last for the lifetime of a SCP (command)
transaction between the host and SpiNNaker. The number of packets which
have passed through the IPTag since it was created is also displayed.

\item
\texttt{iptag <tag.D> set <ip\_addr.P> <port.D>} - this form of the
command sets an IPTag. It is usual for tag 0 to be pre-allocated
to the controlling host with a port number of 17892 (the Tubotron port).
The \texttt{ip\_addr} can be a numeric address or a hostname
in which case it will be looked up using DNS.

The character \texttt{.} can also be supplied for the \texttt{ip\_addr}
in which case the IP address of the host on which \ybug\ is running will
be used. The IP address \texttt{0.0.0.0} will cause the source IP
address of the UDP/IP packet which carries the \texttt{IPTag} command
into SpiNNaker to be used. This is useful where the packet has been
through address translation (eg NAT) en-route to SpiNNaker. Some examples.

\begin{tabular}{p{0.5cm} p{6cm} l}
& \texttt{iptag 3 set 192.168.0.4 2222} & Set IPTag 3, port=2222. IP=192.168.0.4 \\
& \texttt{iptag 2 set . 15555} & Set IPTag 2, port=15555 IP = host IP address \\
\end{tabular}

\item
\texttt{iptag <tag.D> reverse <port.D> <dest\_addr.X> <dest\_port.X>} -
this creates a reverse tag which forwards incoming UDP packets on the
specified port to the SpiNNaker chip specified by \texttt{dest\_addr}
and \texttt{dest\_port}. The UDP payload (which must be small enough
to fit in an SDP packet) is copied into an SDP packet and delivered
within SpiNNaker. When a packet arrives, a reverse path is set up so
that a response can be sent by using the same IPTag.

\begin{tabular}{p{0.5cm} p{6cm} l}
& \texttt{iptag 5 reverse 12345 0304 23} & Set reverse IPTag 5, port=1234
\end{tabular}

The example above will cause incoming UDP packets on port 12345 to be
sent on to chip (3,4), core 3, port 1.

\item
\texttt{iptag <tag.D> clear} - this form of the command clears (removes) an
IPTag table entry.

\end{itemize}

\section{Debugging Commands}

A number of commands are provided specifically to help with debugging
and others are likely to be added on demand.

\begin{itemize}

\item
\texttt{sver} - this command queries the currently addressed core
to find out what operating software it is currently running. This will
usually be SC\&MP for monitor processors (core 0) and SARK for
application processors. The information includes the version number of
the software, the system it is running on (usually SpiNNaker) and the
date it was built. The physical core number of the core is also shown
(in square brackets).

\item
\texttt{led <0123>* on|off|inv|flip } - controls the LEDs attached to
a selected SpiNNaker chip. Up to 4 LEDs can be turned on, off or
inverted.  Some SpiNNaker boards have fewer than 4 LEDs.

\item
\texttt{heap sdram|sysram|system} - displays the allocated and free
blocks in one of the three shared heaps on the selected chip. With no
argument, all three heaps are displayed. The Tag and AppID associated
with each allocated block are shown.

\item
\texttt{iobuf <core.D> [<file.F>]} - dumps the contents of the IO buffer of the
specified core. The IO buffer contains the output
of \texttt{io\_printf} functions which have been executed on the
core. It is assumed that this is ASCII text and it is printed to the
terminal. The IO buffers are allocated in the System Heap and are kept
until the application is terminated with \texttt{app\_stop}. If
the \texttt{file} argument is given, the text is written to the file
rather than the terminal.

\end{itemize}

\section{Commands to control the Router}

These commands control the router on the selected chip. They are
mainly concerned with controlling the MC routing tables.

\begin{itemize}

\item
\texttt{rtr\_load <file.F> <app\_id.D>} - loads the specified file
into the router MC table and associates the given AppID with the set
of entries that is created. It is assumed that the file is in the
appropriate format though some rudimentary checks are performed to
verify that this is the case. The number of entries that have to be
allocated in the router table is determined from the file and a block
of that size is requested. If this is successful then the file is
loaded into the appropriate entries in the table.

\item
\texttt{rtr\_dump} - displays all router MC entries which are
currently allocated. The key, mask and route fields are displayed as
well as the AppID and core associated with each entry. Note that
entries allocated with \texttt{rtr\_load} will show an AppID of
0. Their real AppID is stored in the router MC heap data structures
(see below).  Entries which were initialised by application code
running on SpiNNaker will show the true AppID.

\item
\texttt{rtr\_heap} - displays the router MC heap which lists allocated
and free blocks and the AppID associated with allocated blocks. Note
that the output will only be meaningful if applications have chosen to
use the router MC heap. If they have allocated MC entries manually,
the display may be misleading.

\item
\texttt{rtr\_diag [clr]} - displays the router diagnostic count registers
which were set up when the system booted. This shows the number of
packets of various types which have passed through the router. If the
\texttt{clr} flag is given then the counts are zeroed after the
counts have been displayed. Thus, when the command is given again, the
counts will refer to the period since the clearing command was given.
If an application has modified the configuration of the count
registers then the display is likely to be misleading.

\end{itemize}

\section{Commands to control a Serial ROM}

Some SpiNNaker chips have an SPI-based serial EEPROM (Serial ROM or
SROM) attached to their GPIO port. This can be used to bootstrap the
chip or to hold data which needs to be non-volatile. Chips which have
an Ethernet interface use a Serial ROM to hold MAC and IP addresses.
and \ybug\ allows these parameters to be changed. This is applicable
to Spin2, Spin3 and Spin4 boards. These commands are only accessible
in \texttt{expert} mode.

\begin{itemize}

\item
\texttt{srom\_ip [<ip\_addr.P> [<gw\_addr.P> [<net\_mask.P>]]]} -
displays and/or sets IP addresses in the SROM. Without an argument
this command displays the current IP and MAC addresses. With one
argument it sets the IP address of the system. With a second argument
it also sets the gateway IP address and with a third argument it also
sets the net mask.

Take great care to set these parameters carefully or you may not be
able to communicate with the system after the changes have taken
effect. The SROM is read when the system bootstraps so changes made by
this command will not take effect until the system is reset.

\item
\texttt{srom\_init} - completely initialises an SROM which contains
IP addresses. Use \texttt{srom\_init ?} to get a list of the
arguments. This command is probably not required by most users.

\item
\texttt{srom\_read [<addr.X>]} - reads and displays as a hex dump the
contents of the SROM at the specified address. Useful for debugging
SROM problems but probably not required by most users.

\item
\texttt{srom\_write <file.f> <addr.X>} - writes the contents of the
given file into the SROM at the specified address. Again, one for
experts only and likely to be useful if you want to create your own
custom bootstrap SROM.

\item
\texttt{srom\_erase} - completely erases the SROM (all bytes are set to
\texttt{0xff}). Only do this is you are sure you know what you are
doing!

\end{itemize}

\section{SpiNNaker Board Control Commands}

The 48-node SpiNNaker boards (Spin4 and Spin5) have a Board Management
Processor (BMP) which has its own network interface. The BMP controls
the overall operation of the board and provides facilities to reset
the SpiNNaker chips. A single BMP can control all of the SpiNNaker
boards in a sub-rack.

\ybug\ provides an interface to the BMP for this purpose. The IP
address of the BMP must be specified on the command line with the
\texttt{-bmp} flag. The IP address (or host name) must be followed
by a ``/'' character and then a range of slot numbers which represent
the various SpiNNaker boards which are being used in the current
\ybug\ session. Typically this will just be a single board. For example

\begin{tabular}{p{1cm} p{12cm}}
& \texttt{ybug 192.168.240.25 -bmp 192.168.240.0/3-5} \\
& \texttt{ybug spinn-9 -bmp spinn-9c/0} \\
\end{tabular}

The first command indicates that the controlling BMP is at IP address
\texttt{192.168.240.0} and the SpiNNaker systems that should be
controlled in the \ybug\ session are in slots 3 through 5 of the
sub-rack. The second command indicates that the controlling BMP has
hostname \texttt{spinn-9c} and that the SpiNNaker system that is being
controlled is in slot 0 (ie on the same board as the BMP). This is the
common case of a single Spin4 or Spin5 board.

In the situation where many SpiNNaker boards in a sub-rack are being
used as individual systems, it is important to correctly specify which
boards are being controlled by the BMP. Otherwise, the wrong systems
can end up being reset!

\begin{itemize}

\item
\texttt{reset} - resets all of the specified SpiNNaker systems. This
applies a hard reset to all SpiNNaker chips on each controlled system.
It will be necessary to boot those systems after this command is
given.

\item
\texttt{power on|off} - turns the power to the SpiNNaker systems on
or off. This can be used to put the system into a low power mode when
it is not being used. Turning the power on also applies a reset to the
SpiNNaker chips. Where the system has SpinLink FPGAs (Spin4 and Spin5
boards), these are also put into power-down mode when the power is
turned off. Their configuration is reloaded when they are powered on
again.

\end{itemize}

\rule{\linewidth}{1pt}

\subsection{\itshape Change log:}

\begin{itemize}
\item {\itshape 1.20 - 18aug13 - ST} - initial release - comments to
  {\itshape \Email}
\item {\itshape 1.30 - 02apr14- ST} - update for 1.30
\item {\itshape 1.33 - 19sep14- ST} - update for 1.33 - IPTag changes,
String::CRC32 dependency.
\item {\itshape 2.0.0 - 08mar16- ST} - update for 2.0.0
\end{itemize}

\end{document}
