% Copyright (c) 2009 The University of Manchester
%
% Licensed under the Apache License, Version 2.0 (the "License");
% you may not use this file except in compliance with the License.
% You may obtain a copy of the License at
%
%     http://www.apache.org/licenses/LICENSE-2.0
%
% Unless required by applicable law or agreed to in writing, software
% distributed under the License is distributed on an "AS IS" BASIS,
% WITHOUT WARRANTIES OR CONDITIONS OF ANY KIND, either express or implied.
% See the License for the specific language governing permissions and
% limitations under the License.

\def\FullTitle{\textsl{gdb-spin} - GDB SpiNNaker Interface}
\def\ShortTitle{\textsl{gdb-spin}}
\def\Date{8 Mar 2016}
\def\Version{2.0.0}
\def\Author{Steve Temple}
\def\Email{steven.temple@manchester.ac.uk}

%\def\TitleC{\textcolor{red}{\textbf{Preliminary Draft - Comments Welcome}}}

%
% Simple LaTeX header to generate SpiNNaker documentation
%
% Generates AppNotes if variable NoteNum is defined otherwise
% just generates a simple note. The variables TitleA and TitleB
% may be overridden (second and third lines of title heading)
% TitleC is optional and only used if defined.
%

% Copyright (c) 2009-2019 The University of Manchester
%
% This program is free software: you can redistribute it and/or modify
% it under the terms of the GNU General Public License as published by
% the Free Software Foundation, either version 3 of the License, or
% (at your option) any later version.
%
% This program is distributed in the hope that it will be useful,
% but WITHOUT ANY WARRANTY; without even the implied warranty of
% MERCHANTABILITY or FITNESS FOR A PARTICULAR PURPOSE.  See the
% GNU General Public License for more details.
%
% You should have received a copy of the GNU General Public License
% along with this program.  If not, see <http://www.gnu.org/licenses/>.

\documentclass[11pt,a4paper]{article}

\usepackage{fancyhdr}
\usepackage{color}
\usepackage{ifthen}
\usepackage{xspace}
\usepackage{listings}
\usepackage{enumitem}
\usepackage{alltt}
\renewcommand{\ttdefault}{txtt}
%\usepackage[compact]{titlesec}

\usepackage[pdftex,a4paper]{geometry}
\usepackage[pdftex]{graphicx}

\geometry{left=25mm,right=25mm,top=25mm,bottom=20mm}

\ifdefined\TitleA
\else
  \def\TitleA{SpiNNaker Group, School of Computer Science,
    University of Manchester}
\fi

\ifdefined\TitleB
\else
 \def\TitleB{\Author\ - \Date\ - Version \Version}
\fi

\ifdefined\NoteNum
 \def\HeadLeft{SpiNNaker AppNote \NoteNum}
 \def\HeadCentre{\ShortTitle}
 \def\HeadRight{Page \thepage}
 \def\Title{AppNote \NoteNum\ - \FullTitle}
\else
 \def\HeadLeft{\ShortTitle}
 \def\HeadCentre{}
 \def\HeadRight{Page \thepage}
 \def\Title{\FullTitle}
\fi

\addtolength{\headheight}{2pt}

\pagestyle{fancy}

\renewcommand{\headrulewidth}{0.6pt}

\lhead{\itshape\HeadLeft}
\chead{\itshape\HeadCentre}
\rhead{\itshape\HeadRight}

\fancyfoot{}

\newcommand{\image}[5][]
{
  \begin{figure}[#2]
  \begin{center}
  \includegraphics[#1]{#3}
  \ifthenelse{\equal{#5}{}}{}{\caption{#5}}
  \label{fig:#4}
  \end{center}
  \end{figure}
}

% A substitute for changepage/chngpage's adjustwidth: provides a changemargin
% environment.
\def\changemargin#1#2{\list{}{\rightmargin#2\leftmargin#1}\item[]}
\let\endchangemargin=\endlist 

\newenvironment{shell}
{
  \begin{changemargin}{2em}{0em}
  \begin{alltt}
  \small
}
{
  \end{alltt}
  \end{changemargin}
}

\setcounter{secnumdepth}{-1} 
\setlength{\parindent}{0pt}
\setlength{\parskip}{6pt}

\setitemize{itemsep=0pt}
\setitemize{parsep=0pt}
\setitemize{leftmargin=12pt}

\lstset{xleftmargin=2em,columns=flexible,basicstyle=\ttfamily\small}

\begin{document}

\begin{center}
\setlength{\parskip}{0em}
{\Large\bfseries\Title}
\par
\vspace{5mm}
{\large\itshape\TitleA}
\par
\vspace{2mm}
{\large\itshape\TitleB}
\ifdefined\TitleC
\par
\vspace{2mm}
{\large\itshape\TitleC}
\fi
\end{center}


\newcommand{\gdbspin}{\textsl{gdb-spin}}

\section{Introduction}

This note describes \textbf{\gdbspin}, which is an interface between
GDB (the GNU Debugger) and SpiNNaker. \gdbspin\ is a simple Perl
program which receives debugging commands from GDB and passes them to
a specific core in a SpiNNaker system.

It is currently at a very primitive stage where the only thing it can
do is allow inspection of the static variables of a program executing
on SpiNNaker. Nonetheless, this is still a very useful thing to be
able to do and facilitates debugging programs while they execute
rather than having to stop them as is usual with GDB and similar
debuggers.

You should note that \gdbspin\ is a software debugger and relies on
the target system operating correctly if it is to be functional. If
the target system is crashed, you will need to use a hardware (JTAG)
debugger to do a port-mortem debug.

\section{Principle of Operation}

GDB defines a remote debugging protocol based on packets of ASCII text
which are sent to and from a target system which is being
debugged. Normally, these packets are sent directly to the target
system via a serial line or using TCP/IP. The packets typically
contain requests to read or write memory, set breakpoints or start and
stop execution on the target system. In order for this to work, a
GDB \textbf{debug stub}, which responds to the debug packets, is
normally installed as part of the target application software.

Building a GDB debug stub is quite difficult to do in SpiNNaker and
so \gdbspin\ uses a different strategy. GDB connects (via TCP/IP)
to \gdbspin\ which runs on a host workstation. Debug packets from GDB
are translated by \gdbspin\ which then forwards them to a specific
core on the target SpiNNaker system as SCP (SpiNNaker Command
Protocol) packets. In its present form, \gdbspin\ only forwards memory
read requests which allows GDB to interrogate the memory on the
selected core.

In order for GDB to know the position of variables in memory, it needs
to have debug information generated during the SpiNNaker application
build process. This will normally be contained in the ELF file
generated in the build. It's important to note that the compilation
(and assembly, if appropriate) steps need to be told to generate this
information by giving them the \texttt{-g} (debug) flag when they are
run. All libraries must also be (re)generated with this flag if their
internal variables are to be visible.

\section{Running \gdbspin}

The essential steps in getting \gdbspin\ to work are to install it and
start it running and then to start GDB and instruct GDB to connect
to \gdbspin. It's also necessary to be able to tell \gdbspin\ which
SpiNNaker core is being debugged and there is a simple interface,
accessible from GDB, to do this.

\gdbspin\ uses the Perl libraries which come with the SpiNNaker
low-level software tools to connect to SpiNNaker. If you have these
tools installed then you should ensure that \gdbspin\ is in
the \texttt{.../tools/} sub-directory (and is executable). You should
then be able to start it from the command line like this

\begin{shell}
unix> \textbf{gdb-spin <spin\_ip> [<port>]}
\end{shell}

You need to specify the IP address or hostname of the SpiNNaker system
(\texttt{spin\_ip}) and there is an optional TCP/IP port number which
is the port on which it listens for requests from GDB. This defaults
to 17899 and will probably only ever need to be changed if you want to
run multiple instances of \gdbspin\ on the same host. If all is well
when \gdbspin\ starts you should see several lines of information
displayed on the terminal. There is currently no friendly way to
terminate \gdbspin\ other than typing CTRL/C in its terminal.

\subsection{\gdbspin\ commands}

\gdbspin\ accepts commands from the user via the GDB command
line. At present there are only two commands -
\texttt{sp} which selects which SpiNNaker core to debug and
\texttt{debug} which controls the level of debugging output from
\gdbspin\ and is unlikely to be useful to most people.
\texttt{sp} is similar to the same command in \texttt{ybug}, taking
up to three numeric arguments to select a core on a SpiNNaker chip
for debugging.

To issue a command from the GDB command line you need to prefix it
with GDB's \texttt{monitor} command as below. This instructs GDB to
pass the command directly to \gdbspin. The command \texttt{sp} without
arguments will display the currently selected core. For example

\begin{shell}
(gdb) \textbf{monitor sp}
#  (1) Current core 0 0 0
(gdb) \textbf{monitor sp 2 5 7}
#  (1) Select core 2 5 7
\end{shell}

Note that core (0, 0, 0) is selected by default when \gdbspin\
starts.

\subsection{Multiple sessions}

\gdbspin\ supports many (up to 25) simultaneous GDB connections
so that you can be debugging many cores at the same time. Each new
connection is allocated an identifier which is an integer which
increments for each new connection. This identifier is displayed
(in parentheses) in all output from \gdbspin.

\section{Running GDB}

Pretty much any (recent-ish) version of GDB will talk to \gdbspin\ and
allow you to perform debugging.  You may want to use the one which
comes with the GNU software tools you are using to build SpiNNaker
applications. This will (probably) be \texttt{arm-none-eabi-gdb}.
Note that applications built with other tool chains (eg ARM
compiler/assembler) can be also debugged provided that they generate
ELF files in the usual format. The description below refers to
command-line based debugging but there is no reason that GUI-based
front ends to GDB would not work.

GDB can be started without arguments or you can specify the name of
the ELF file that you wish to debug. You will see a
prompt \texttt{(gdb)} and you can then issue debugging commands. The
first thing to do is to connect to the target system
(via \gdbspin). This is done with the \texttt{target remote} command
giving the host and port on which \gdbspin\ is running.  In most
cases, this will be the same machine as that on which GDB is running
in which case the host can be omitted or specified
as \texttt{localhost}. In the examples below, the first two are
equivalent while the third debugs using an instance of
\gdbspin\ on a remote machine.

\begin{shell}
unix> \textbf{arm-none-eabi-gdb gdb-test.elf}
GNU gdb (Sourcery CodeBench Lite 2013.05-23), etc, etc
(gdb) \textbf{target remote :17899}
(gdb) \textbf{target remote localhost:17899}
(gdb) \textbf{target remote weasel.cs.man.ac.uk:17899}
\end{shell}

At this point, GDB can issue debug requests to \gdbspin\ which will
then forward them to SpiNNaker. To leave GDB, you need to give
the \texttt{quit} command and this will close the connection to
\gdbspin.

\section{A Simple Example}

A simple example showing the use of GDB and \gdbspin\ will now be
described. The program to be debugged is shown below. It is a simple
API application which creates a 10ms timer whose callback just updates
some static variables which can then be inspected with GDB.

\begin{lstlisting}[language=C]

// Simple SpiNNaker application to demonstrate use of GDB

#include <spin1_api.h>

// Declare a struct type to contain data for this program

typedef struct my_data
{
  uint count;
  uchar list[16];
  struct my_data *self;
} my_data_t;

// Declare an instance of the data struct. This is what we want to inspect
// with GDB. Placing many or all of the module's variables in a single
// struct is useful for debugging as we can display them all with a single
// command in GDB. The compiler is just as efficient at addressing these
// variables whether or not they are located in a struct.

my_data_t app_data;

// A timer function, called every 10ms. This copies the tick count into
// the data struct and updates one of the array elements with a new
// random number each time. It also sets the 'self' pointer to itself
// so that we can see this in GDB.

void timer_proc (uint ticks, uint arg2)
{
  app_data.count = ticks;
  app_data.list[ticks % 16] = spin1_rand ();
  app_data.self = &app_data;
}

// Main function which sets up and runs the timer

void c_main (void)
{
  spin1_callback_on (TIMER_TICK, timer_proc, 1);
  spin1_set_timer_tick (10 * 1000);
  spin1_start (SYNC_NOWAIT);
}
\end{lstlisting}

The program source is \texttt{gdb-test.c} and it is compiled to an APLX
file and loaded onto a SpiNNaker core with \texttt{ybug} using the
following commands. Note the use of the \texttt{CFLAGS} parameter in
the \texttt{make} command which causes debug information to be placed
in the ELF file which is also created by the make process.

\begin{shell}
unix> \textbf{make APP=gdb-test CFLAGS=-g}
unix> \textbf{ybug 192.168.240.38}
192.168.240.38:0,0,0 > \textbf{app_load gdb-test.aplx . 1 16}
192.168.240.38:0,0,0 > \textbf{ps}
Core State  Application       ID   Running  Started
---- -----  -----------       --   -------  -------
  0  RUN    scamp-134          0   0:12:07   6 Jan 12:05
  1  RUN    gdb-test          16   0:00:03   6 Jan 12:17
  2  IDLE   sark               0   0:12:07   6 Jan 12:05
  3  IDLE   sark               0   0:12:07   6 Jan 12:05
\end{shell}

At this point the program is loaded and running on core 1 of chip (0,
0) of a SpiNNaker system whose IP address
is \texttt{192.168.240.38}. We can now use GDB to inspect its state.
First we start \gdbspin\ telling it to talk to \texttt{192.168.240.38}.
This will need to be done in a new shell (as will the subsequent GDB
session).

\begin{shell}
unix> \textbf{gdb-spin 192.168.240.38}
#-------------------------------------------------------------------------------
#
# SpiNNaker GDB Interface - Version 0.01 (experimental!)
#
# Connecting to SpiNNaker   192.168.240.38
# Starting GDB interface    0.0.0.0:17899
#
#-------------------------------------------------------------------------------
#
\end{shell}

Next we start GDB, instruct it to connect to \gdbspin\ and load the
relevant ELF file. We also give a command to make the printing of
structs a bit nicer. Finally, we issue a command to \gdbspin\ to
select the core on which our program is running.

\begin{shell}
unix> \textbf{arm-none-eabi-gdb }
GNU gdb (Sourcery CodeBench Lite 2013.05-23), etc, etc
(gdb) \textbf{target remote :17899}
(gdb) \textbf{symbol-file gdb-test.elf}
Reading symbols from /tmp/spinnaker_tools_134/apps/gdb/gdb-test.elf
(gdb) \textbf{set print pretty on}
(gdb) \textbf{monitor sp 0 0 1}
#  (1) Select core 0 0 1
\end{shell}

GDB is now all set up to interact with the target core. To inspect a
variable you use the \texttt{print} command. This will cause GDB to
read the variable from SpiNNaker memory and print it on the
terminal. The example below reads the \texttt{app\_data} variable
three times over a period of a few seconds so that the change in the
variable's value is obvious. Adding the \texttt{/x} qualifier
to \texttt{print} causes output in hexadecimal. The fourth command
reads a single element of the struct.

\begin{shell}
(gdb) \textbf{print app_data}
$1 = {
  count = 1836203,
  list = "\verb=\=b\verb=\=350\verb=\=255*\verb=\=035:\verb=\=212#r\verb=\=262\verb=\=217\verb=\=033\verb=\=301",
  self = 0x400360 <app_data>
}
(gdb) \textbf{print/x app_data}
$2 = {
  count = 0x1c06d0,
  list = {0x3b, 0x73, 0xfe, 0xb5, 0x28, 0x97, 0xb4, 0x4c, 0x50, 0x8b, 0xfe,
    0xe3, 0x36, 0x6e, 0x62, 0xc3},
  self = 0x400360
}
(gdb) \textbf{print/x app_data}
$3 = {
  count = 0x1c08b0,
  list = {0xac, 0x4f, 0x7e, 0x30, 0xa3, 0x38, 0x8a, 0x57, 0x23, 0xf, 0x1b,
    0xf6, 0x63, 0x81, 0xd6, 0x5a},
  self = 0x400360
}
(gdb) \textbf{print app_data.self}
$4 = (struct my_data *) 0x400360 <app_data>
\end{shell}

You should note that if you change the program and recompile and
reload it, it's possible that the ELF file you have loaded in GDB will
become inconsistent with the newly loaded code and so it's wise to
reload the ELF (using the \texttt{symbol-file} command) each time you
do this.

The GDB command line interpreter performs command and variable name
auto-completion and maintains a command history so is quite easy to
use.

\section{Current Limitations}

The main limitation of the current implementation of \gdbspin\ is
that it can only read and display static variables. In order to safely
read variables on the stack the program must be stopped (so that the
stack pointer is not moving around). This is almost certain to be
fatal for most SpiNNaker programs which have real-time constraints.
If there is no requirement for the program to continue after it
has been stopped then it would probably be possible to provide this
facility.

Similarly, providing a breakpoint mechanism would be possible but this
would also stop program execution and so would only be useful in a
limited range of circumstances. I would be happy to look at this if it
was thought to be useful.

Finally, because GDB/\gdbspin\ needs to read SpiNNaker memory and
needs the application processor to be up and running to do this, none
of this will work if the application core has crashed badly and is
unable to respond to SCP requests from the monitor processor.

\rule{\linewidth}{1pt}

\subsection{\itshape Change log:}

\begin{itemize}
\item {\itshape 0.01 - 05jan15 - ST} - preliminary draft - comments to
  {\itshape \Email}
\item {\itshape 2.0.0 - 08mar16 - ST} - first release
\end{itemize}

\end{document}
