% Copyright (c) 2009-2019 The University of Manchester
%
% This program is free software: you can redistribute it and/or modify
% it under the terms of the GNU General Public License as published by
% the Free Software Foundation, either version 3 of the License, or
% (at your option) any later version.
%
% This program is distributed in the hope that it will be useful,
% but WITHOUT ANY WARRANTY; without even the implied warranty of
% MERCHANTABILITY or FITNESS FOR A PARTICULAR PURPOSE.  See the
% GNU General Public License for more details.
%
% You should have received a copy of the GNU General Public License
% along with this program.  If not, see <http://www.gnu.org/licenses/>.

\def\FullTitle{SARK - SpiNNaker Application Runtime Kernel}
\def\ShortTitle{SARK 2.0.0}
\def\Date{8 Mar 2016}
\def\Version{2.0.0}
\def\Author{Steve Temple}
\def\Email{steven.temple@manchester.ac.uk}

%\def\TitleB{\textcolor{red}{\textbf{Preliminary Draft - Comments Welcome}}}

%
% Simple LaTeX header to generate SpiNNaker documentation
%
% Generates AppNotes if variable NoteNum is defined otherwise
% just generates a simple note. The variables TitleA and TitleB
% may be overridden (second and third lines of title heading)
% TitleC is optional and only used if defined.
%

% Copyright (c) 2009-2019 The University of Manchester
%
% This program is free software: you can redistribute it and/or modify
% it under the terms of the GNU General Public License as published by
% the Free Software Foundation, either version 3 of the License, or
% (at your option) any later version.
%
% This program is distributed in the hope that it will be useful,
% but WITHOUT ANY WARRANTY; without even the implied warranty of
% MERCHANTABILITY or FITNESS FOR A PARTICULAR PURPOSE.  See the
% GNU General Public License for more details.
%
% You should have received a copy of the GNU General Public License
% along with this program.  If not, see <http://www.gnu.org/licenses/>.

\documentclass[11pt,a4paper]{article}

\usepackage{fancyhdr}
\usepackage{color}
\usepackage{ifthen}
\usepackage{xspace}
\usepackage{listings}
\usepackage{enumitem}
\usepackage{alltt}
\renewcommand{\ttdefault}{txtt}
%\usepackage[compact]{titlesec}

\usepackage[pdftex,a4paper]{geometry}
\usepackage[pdftex]{graphicx}

\geometry{left=25mm,right=25mm,top=25mm,bottom=20mm}

\ifdefined\TitleA
\else
  \def\TitleA{SpiNNaker Group, School of Computer Science,
    University of Manchester}
\fi

\ifdefined\TitleB
\else
 \def\TitleB{\Author\ - \Date\ - Version \Version}
\fi

\ifdefined\NoteNum
 \def\HeadLeft{SpiNNaker AppNote \NoteNum}
 \def\HeadCentre{\ShortTitle}
 \def\HeadRight{Page \thepage}
 \def\Title{AppNote \NoteNum\ - \FullTitle}
\else
 \def\HeadLeft{\ShortTitle}
 \def\HeadCentre{}
 \def\HeadRight{Page \thepage}
 \def\Title{\FullTitle}
\fi

\addtolength{\headheight}{2pt}

\pagestyle{fancy}

\renewcommand{\headrulewidth}{0.6pt}

\lhead{\itshape\HeadLeft}
\chead{\itshape\HeadCentre}
\rhead{\itshape\HeadRight}

\fancyfoot{}

\newcommand{\image}[5][]
{
  \begin{figure}[#2]
  \begin{center}
  \includegraphics[#1]{#3}
  \ifthenelse{\equal{#5}{}}{}{\caption{#5}}
  \label{fig:#4}
  \end{center}
  \end{figure}
}

% A substitute for changepage/chngpage's adjustwidth: provides a changemargin
% environment.
\def\changemargin#1#2{\list{}{\rightmargin#2\leftmargin#1}\item[]}
\let\endchangemargin=\endlist 

\newenvironment{shell}
{
  \begin{changemargin}{2em}{0em}
  \begin{alltt}
  \small
}
{
  \end{alltt}
  \end{changemargin}
}

\setcounter{secnumdepth}{-1} 
\setlength{\parindent}{0pt}
\setlength{\parskip}{6pt}

\setitemize{itemsep=0pt}
\setitemize{parsep=0pt}
\setitemize{leftmargin=12pt}

\lstset{xleftmargin=2em,columns=flexible,basicstyle=\ttfamily\small}

\begin{document}

\begin{center}
\setlength{\parskip}{0em}
{\Large\bfseries\Title}
\par
\vspace{5mm}
{\large\itshape\TitleA}
\par
\vspace{2mm}
{\large\itshape\TitleB}
\ifdefined\TitleC
\par
\vspace{2mm}
{\large\itshape\TitleC}
\fi
\end{center}


\section{Introduction}

SARK is the lowest level of software which runs on a SpiNNaker core
(CPU). It is linked together with the application code and performs
three main functions.

\begin{itemize}
\item
It initialises the ARM core, setting up stacks and setting various
control registers in the core and some of its peripherals. It then
calls the main procedure of the application to start the application
running.

\item
It provides a library of low-level functions for the application
which perform operations such as memory management, interrupt control,
etc.

\item
It provides a mechanism for the monitor processor to communicate with
the application core, mostly using SDP packets. This allows the
application's memory to be read and written from a host, provides a
basis for simple I/O functions and allows the application to be
controlled from the host.
\end{itemize}

\section{Application Loading and Start-up}

A SpiNNaker application is a program which runs on a SpiNNaker core
and has sole use of that core. The same application may be loaded into
many cores. It will have been built with a cross-compiler, linked with
various libraries, including SARK, and converted into a format known
as an APLX file ready to be loaded onto a SpiNNaker core. Typically,
the APLX file is loaded into an area of shared memory on the SpiNNaker
chip by the monitor processor which then causes the relevant
application cores to load and start the application. The loading of
the application results in the executable part of the APLX file being
copied to the bottom of ITCM on the core and the data part of the APLX
file being used to initialise the application's data which is placed
in DTCM. Finally, the application core starts executing the
application by branching to address zero where the application's
executable code has been placed. The code at this address is the SARK
start-up code.

\section{Execution Environment}

As the ARM968 cores used in SpiNNaker have a Harvard architecture, two
blocks of local memory (ITCM and DTCM) are provided in SpiNNaker to
take advantage of this and SARK loads application programs into these
memories. Executable code is placed at the bottom of ITCM and the data
variables used by this code are placed in DTCM. The cores also have
access to other memories which are shared by all cores. The executable
code in ITCM occupies a contiguous block at the bottom of the memory
and there is usually an unused portion of memory above this unless the
application fully occupies ITCM. There is a reserved area of 256 bytes
at the top of ITCM which is used for application loading and other
system functions. Figure 1 shows the layout of ITCM and DTCM.

\image[scale=0.6]{ht}{tcm.pdf}{tcm}{Layout of ITCM and DTCM memories}

The application's program variables are stored in DTCM. As most
applications are currently written in C, the allocation of DTCM is
geared towards the standard C data memory usage. Two types of C static
variables are stored at the bottom of DTCM. At the very bottom are
those static variables which are explicitly initialised in the C
source. Their values are copied out of the APLX file just before the
application starts. Above these initialised variables, which are known
as the RW variables, are all the remaining static variables which are
typically uninitialised scalars, structs and arrays. These are known
as the ZI variables and they are all set to zero before the
application starts.

At the top of DTCM are four stacks for four modes of the ARM968 core.
The stack pointers for these modes are initialised by SARK at start-up.
The SVC mode stack is currently only used when the application starts
but is also available for any system calls that may be implemented
using the SVC instruction. There are two stacks for the two interrupt
modes (IRQ and FIQ) which are used by SpiNNaker applications. The
fourth stack is for System mode and this is the mode in which SpiNNaker
applications run. The stacks hold the automatic C variables (ie those
which are declared inside functions) as well as other information such
as return addresses and saved registers.

The area between the top of the ZI variables and the bottom of the
stacks is initialised by SARK for use as a heap which can be accessed
with malloc and free functions which are provided by SARK. The size of
the heap will depend on the size of the RW and ZI data sections. The
location of the various data areas (RW, ZI, heap) and the sizes and
position of the stacks can be changed in specific circumstances but
the defaults shown here should be adequate for most applications.

Access to ITCM and DTCM from the ARM core is fast, usually only
requiring a single CPU clock cycle (typically 5ns). Data and
instructions should be kept in these memories where possible as access
to shared memories (System RAM and SDRAM) is typically 20 to 30 times
slower.

\section{SARK Start-up}

The SARK start-up code is placed in ITCM at address zero and is
entered there to start the application. The overall aim of the
start-up code is to establish the environment before the main
procedure of the application, \texttt{c\_main}, is called. This involves a
significant number of steps, not all of which will be described here.

The start-up code contains the ARM exception vectors and these are all
initialised to point to suitable handler routines. The start-up begins
at the reset vector and then proceeds to do the following steps

\begin{itemize}
\item
Replaces the reset vector with an error handler to trap branches
through zero.

\item
Sets the ARM968 CP15 (coprocessor) control register to enable the
write buffer, trap misaligned data accesses, enable Thumb code, and
use low vectors.

\item
Calls a function \texttt{sark\_config} to allow the application to
adjust stack positions and size and set some other configuration
values before the main SARK initialisation takes place. This function
is declared \textbf{weak} so if the application does not define it, an
empty function is called instead.

\item
Fills the stack areas with a predefined data value so that it's
possible to see how much stack has been used.

\item
Calls a function \texttt{sark\_init} to set up the stacks for the
various modes and set up heap and SDP message buffers. An area of
shared system RAM which allows the application to communicate with the
monitor processor is also set up and the VIC interrupt controller is
initialised. \texttt{sark\_init} is also weak and so can be superseded
by the application but this is unlikely to be necessary in most cases.
On return from \texttt{sark\_init} the core is placed into System mode
with (both) interrupts enabled which is the mode in which applications
execute.

\item
Calls a function \texttt{sark\_wait} to optionally wait for a signal
from the host before proceeding. The wait is done in sleep mode to
minimise power consumption.

\item
Calls a (weak) function \texttt{sark\_pre\_main} to allow the
application to set up anything that needs to be initialised before the
application starts to execute. This is typically used to transparently
set up system infrastructure such as the Spin1 API.

\item
Calls the function \texttt{c\_main} which must be defined by the
application and causes the application to start execution. If
\texttt{c\_main} returns, the following steps also take place.

\item
Calls a (weak) function \texttt{sark\_post\_main} which may be used to
tidy up after the application has finished. Again, this is typically
used to clean up system infrastructure such as that provided by Spin1
API.

\item
Calls the function \texttt{sark\_sleep} which causes the core to go
into a low power mode. In this mode, interrupts are still serviced so
it will still be possible for the monitor processor to talk to the
core.

\end{itemize}

\section{SARK Source Files}

SARK is written mostly in C with a single assembly language source
file which contains the start-up code and a number of hand-coded
library routines. Two C header files are required and the full list
is as follows

\begin{lstlisting}
spinnaker.h         # SpiNNaker hardware definitions
sark.h              # SARK definitions and documentation

sark_alib.s         # Assembly language bits and pieces

sark_alloc.c        # Memory and router allocation
sark_base.c         # SDP message passing and some basic functions
sark_event.c        # Event handling
sark_hw.c           # Routines to interface to SpiNNaker hardware
sark_io.c           # A simple IO library providing "printf"
sark_isr.c          # Interrupt service routines
sark_pkt.c          # Packet transmission routines
sark_timer.c        # Routines providing timed events
\end{lstlisting}

To build an application using SARK it is necessary to include the
header file \texttt{sark.h} in your C source. This will also
automatically include \texttt{spinnaker.h}.

\texttt{sark.h} is commented in Doxygen style to allow automatic
generation of documentation and this file should be used as the
primary reference for documentation of SARK's types, data structures
and functions.

\section{SARK Library Functions}

SARK provides a set of low-level functions which may be used by
application programs or by other system-level software. They are
documented in \texttt{sark.h} and the following list is only intended
to give a general indication of what is provided. The functions fall
into the following categories.

\begin{itemize}

\item
Control of the ARM968 CPU - disabling and enabling interrupts, writing
and reading the CP15 control register and the CPSR, putting the CPU
into sleep mode, shutting down the CPU.

\lstset{language=C}
\begin{lstlisting}
uint cpu_fiq_disable (void);
uint cpu_fiq_enable (void);
uint cpu_int_disable (void);
uint cpu_int_enable (void);
uint cpu_irq_disable (void);
uint cpu_irq_enable (void);

void cpu_int_restore (uint cpsr);

uint cpu_get_cpsr (void);
void cpu_set_cpsr (uint cpsr);

uint cpu_get_cp15_cr (void);
void cpu_set_cp15_cr (uint value);

void cpu_wfi (void);
void cpu_sleep (void) __attribute__((noreturn));
void cpu_shutdown (void) __attribute__((noreturn));
\end{lstlisting}

\item
Memory manipulation - simple routines to copy and fill memory which
are generally smaller (but less efficient) than the corresponding ARM
library routines.

\lstset{language=C}
\begin{lstlisting}
void sark_mem_cpy (void *dest, const void *src, uint n);
void sark_str_cpy (char *dest, const char *src);
uint sark_str_len (char *string);
void sark_msg_copy (sdp_msg_t *to, sdp_msg_t *from);
void sark_word_cpy (void *dest, const void *src, uint n);
void sark_word_set (void *dest, uint data, uint n);
\end{lstlisting}

\item
Random number generation - a simple pseudo-random number generator for
32-bit numbers with a seeding function.

\lstset{language=C}
\begin{lstlisting}
void sark_srand (uint seed);
uint sark_rand (void);
\end{lstlisting}

\item
SDP messaging - allocation of buffers in DTCM and shared memory for
SDP messages and a routine to send an SDP message via the monitor
processor.

\lstset{language=C}
\begin{lstlisting}
sdp_msg_t *sark_msg_get (void);
void sark_msg_free (sdp_msg_t *msg);
sdp_msg_t *sark_shmsg_get (void);
void sark_shmsg_free (sdp_msg_t *msg);
uint sark_msg_send (sdp_msg_t *msg, uint timeout);
\end{lstlisting}

\item
Error handling routines to allow an application to signal that it has
encountered an error.

\lstset{language=C}
\begin{lstlisting}
void rt_error (uint code, ...);
void sw_error (sw_err_mode mode);
\end{lstlisting}

\item
Simple text output - a primitive \texttt{printf} which sends text
output via SDP, to an SDRAM buffer or to a string
(like \texttt{sprintf}).

\lstset{language=C}
\begin{lstlisting}
void io_printf (char *stream, char *format, ...);
\end{lstlisting}

\item
Locks and semaphores - routines to acquire and free one of the
32 hardware locks provided by the chip. Routines to lower and
raise arbitrary 8-bit semaphore variables using one of the
hardware locks to control access. Routines to lower and raise
application-specific semaphores and to count the number of cores
running an application.

\lstset{language=C}
\begin{lstlisting}
uint sark_lock_get (spin_lock lock);
void sark_lock_free (uint cpsr, spin_lock lock);

void sark_sema_raise (uchar *sema);
uint sark_sema_lower (uchar *sema);

uint sark_app_raise (void);
uint sark_app_lower (void);
uint sark_app_sema (void);

uint sark_app_lead (void);
uint sark_app_cores (void);
\end{lstlisting}

\item
Memory management - malloc and free routines for memory in DTCM,
System RAM and SDRAM. Routines to allocate and free blocks of router
MC table entries.

\lstset{language=C}
\begin{lstlisting}
void *sark_alloc (uint count, uint size);
void sark_free (void *ptr);

void *sark_xalloc (heap_t *heap, uint size, uint id, uint lock);
void sark_xfree (heap_t *heap, void *ptr, uint lock);
uint sark_xfree_id (heap_t *heap, uint id, uint lock);

void *sark_tag_ptr (uint tag, uint app_id);
uint sark_heap_max (heap_t *heap, uint lock);
heap_t *sark_heap_init (uint *base, uint *top);

uint rtr_alloc (uint size);
uint rtr_alloc_id (uint size, uint app_id);
void rtr_free (uint entry, uint clear);
uint rtr_free_id (uint id, uint clear);
\end{lstlisting}

\item
Environment queries - routines to get core ID, chip ID, application
ID and name.

\lstset{language=C}
\begin{lstlisting}
uint sark_core_id (void);
uint sark_chip_id (void);
uint sark_app_id (void);
char *sark_app_name (void);
\end{lstlisting}

\item
Hardware management - routines to control the VIC (vectored interrupt
controller). Routines to control any LEDs attached to the SpiNNaker
chip. A routine to busy-wait for a number of microseconds. Routines to
initialise and control the various tables in the router.

\lstset{language=C}
\begin{lstlisting}
void sark_vic_init (void);
void sark_vic_set (vic_slot slot, uint interrupt, uint enable,
                     int_handler handler);

void sark_led_init (void);
void sark_led_set (uint leds);

void sark_delay_us (uint n);

void rtr_mc_init (void);
uint rtr_mc_clear (uint start, uint count);
void rtr_mc_set (uint entry, uint key, uint mask, uint route);
uint rtr_mc_get (uint entry, rtr_entry_t *r);
uint rtr_mc_load (rtr_entry_t *e, uint count, uint offset, uint app_id);

void rtr_p2p_init (void);
void rtr_p2p_set (uint entry, uint value);
uint rtr_p2p_get (uint entry);

void rtr_diag_init (const uint *table);
void rtr_init (uint monitor);
\end{lstlisting}

\item
Timer management - routines to schedule (and cancel) events at some
time in the future to microsecond resolution.

\lstset{language=C}
\begin{lstlisting}
void event_register_timer (vic_slot slot);

void timer_schedule (event_t *e, uint time);
uint timer_schedule_proc (event_proc proc, uint arg1, uint arg2, uint time);
void timer_cancel (event_t *e, uint ID);
\end{lstlisting}

\item
Packet transmission - routines to transmit packets.

\lstset{language=C}
\begin{lstlisting}
void event_register_pkt (uint queue_size, vic_slot slot);

uint pkt_tx_k (uint key);
uint pkt_tx_kd (uint key, uint data);
uint pkt_tx_kc (uint key, uint ctrl);
uint pkt_tx_kdc (uint key, uint data, uint ctrl);
\end{lstlisting}

\item
Event management - Registration of event handlers, routines to start
and stop event processing, routines to add events to event queues.

\lstset{language=C}
\begin{lstlisting}
void event_register_int (event_proc proc, event_type event, vic_slot slot);
void event_register_queue (event_proc proc, event_type event, vic_slot slot,
			   event_priority priority);
void event_register_pause (event_proc proc, uint arg2);

void event_enable (event_type event, uint enable);

uint event_start (uint period, uint events, uint wait);
void event_stop (uint rc);
void event_wait (void);
void event_run (uint restart);
void event_pause (uint pause);

event_t* event_new (event_proc proc, uint arg1, uint arg2);
void event_alloc (uint events);

uint event_queue (event_t *e, event_priority priority);
uint event_queue_proc (event_proc proc, uint arg1, uint arg2,
		       event_priority priority);

uint event_user (uint arg1, uint arg2);
\end{lstlisting}

\end{itemize}

\section{SARK Data Structures}

An application which has been linked with SARK has access to a number
of data structures which contain information relevant to the
application. Some of these data structures are in core-local memories
(ITCM and DTCM) while some are in shared memories (System RAM and
SDRAM). They are all documented in the SARK header files
- \texttt{spinnaker.h} and (primarily) \texttt{sark.h}. A brief
description follows and you should refer to the header files for full
details of the data structures.

A struct of type \texttt{sark\_vec\_t} is located in ITCM at address
0x20. A pointer variable \texttt{sark\_vec} provides access to its
fields. This data structure is intended to be changed only
occasionally. It contains vector addresses for the various ARM
exceptions, specifications of the location and sizes of the ARM
stacks, sizes of various buffers and pointers into various parts of
the DTCM such as the heap base and stack limits. It is built into the
application code and is intended to be used during application
start-up to configure the system. It can be modified during
application start-up (via function \texttt{sark\_config}) to change
some configuration information and as the application runs to change
the exception vectors.

A struct of type \texttt{sark\_data\_t} is located in DTCM at a
link-time dependent address. It may be accessed using the
variable \texttt{sark}.  This data structure contains various pieces
of state which must be accessed as efficiently as possible. It
contains several pointers to other data structures associated with the
application such as the heap (in DTCM) and core-specific buffers in
System RAM and SDRAM.

A struct of type \texttt{event\_data\_t} is also located in DTCM at a
link-time dependent address. It may be accessed using the
variable \texttt{event}. It contains state relating to the SARK event
handling routines and will not be present if this facility is not
being used.

A struct of type \texttt{vcpu\_t} is located in System RAM. A pointer
to this structure is available in the variable \texttt{sark.vcpu}.
This structure contains data which must be accessible to other cores
in the chip, primarily the monitor processor (and hence the
host). Some fields are used for message passing between cores. There
is a register dump field where debug information is placed when the
core encounters a fatal error and also information about what
application is running on the core, what its current state is, when it
started execution, etc. Most of this data structure is reinitialised
every time a new application starts on the core but there are 4 words
(\texttt{user0-user3}) which are not and may be used to pass data
between successive applications starting on the core.

Each core also has a private buffer in SDRAM which is unstructured and
of size 128KB (the size is configurable at system boot time). There is
a pointer to it in the variable \texttt{sark.sdram\_buf}.

\subsection{System Variables}

In addition to the per-core data structures there are a number of
shared data structures such as heaps in the SDRAM and System RAM and
an area of system variables in System RAM. These are mostly set up and
maintained by the monitor processor. The system variables are of
interest to applications because they provide much information about
the state of the system and provide pointers into various pieces of
shared state. The system variables are described by a struct of type
\texttt{sv\_t} and this is mapped into System RAM and accessible via
a pointer variable \texttt{sv}. The system variables are documented
inline in the definition of \texttt{sv\_t} in \texttt{sark.h}.

\section{Shared Memories}

Each SpiNNaker chip contains three memories which are accessible by
all cores over a shared system bus. One of these is the boot ROM from
which the chip bootstraps when it is reset. It is read-only and will
not be discussed further here. The other two memories are an on-chip
32KB RAM memory known as the System RAM and a separate 128MB
synchronous DRAM chip in the same package known as the SDRAM.

The System RAM (aka SysRAM) and SDRAM are extensively used where data
needs to be shared between cores. They are accessible either by normal
load and store operations from the cores or via the DMA controller in
each core which can be used to copy data to or from the local DTCM and
ITCM memories in each core. SARK (and SC\&MP) impose some structure on
the SysRAM and SDRAM which is described here.

\subsection{System RAM}

The organisation of the SysRAM is shown in figure 2. The names to the
left of the diagram indicate the system variable which points to the
relevant part of the memory. The addresses to the right are for
illustration only and should not be used directly by any user code (ie
they may change!).

\image[scale=0.75]{ht}{sysram.pdf}{sysram}{Layout of System RAM memory}

At the bottom of the SysRAM is a 256 byte Reset Block which is
reserved for system use and contains reset vectors and initialisation
code to allow application cores to be reset by the monitor processor
and then to enter user-specified applications. Applications should not
try to alter this area.

Above the Reset Block is an area, the User SysRAM, reserved for user
applications. Its size is configurable when the system boots. There is
a pointer in the system variables which points to the base of this
area. The default configuration sets the size of this area to 4K
bytes.

Above the User SysRAM is the SysRAM Heap, a heap from which memory may
be allocated and freed by calling the appropriate SARK functions. The
size of this area is determined by the amount of User SysRAM which is
allocated. It is used to provide shared memory SDP message buffers for
communication between the monitor processor and application cores.
With the default User SysRAM allocation of 4KB, the heap is around
20KB. The system variable \texttt{sv->sysram\_heap} points to the heap.

Above the SysRAM Heap is an area containing 18 data structures (VCPU
blocks) of type \texttt{vcpu\_t}, one for each core on the chip. These
are used for a variety of actions including message passing, status
saving, etc, etc. The local state of each core contains a pointer
(\texttt{sark.vcpu}) to the relevant VCPU block so that it can be
readily accessed from application code.

Above the VCPU blocks is an area of System Data which is used by
SC\&MP and which should not be directly accessed by application code.

Finally, at the top of SysRAM is a 256 byte area which contains most
of the system variables which are needed by SARK and SC\&MP. This area
is mapped onto a struct of type \texttt{sv\_t} and the fields of the
struct are accessible via the variable \texttt{sv}.

\subsection{SDRAM}

The organisation of the SDRAM is shown in figure 3. As for the SysRAM,
the names to the left of the diagram indicate the system variable
which points to the relevant part of the memory while the addresses to
the right are for illustration only.

\image[scale=0.75]{ht}{sdram.pdf}{sdram}{Layout of SDRAM memory}

At the bottom of the SDRAM is an area containing 18 SDRAM Buffers,
one for each core. The size of these buffers is configurable at boot
time and defaults to 128KB (each). The local state of each core
contains a pointer (\texttt{sark.sdram\_buf}) to the buffer belonging
to that core.

Above the SDRAM Buffers is the User SDRAM, reserved for user
applications. Like the SysRAM, the size is configurable at boot
time. The default configuration is for 4M bytes and there is a pointer
in the system variables to the base of this area.

Above the User SDRAM is the SDRAM Heap from which memory may be
allocated and freed with the appropriate functions. The size is
determined by the amount of memory allocated to the SDRAM Buffers and
User SDRAM and the default allocation for these results in
a heap of around 114M bytes.

Above the SDRAM Heap is a System Buffer which is used by the monitor
processor. At the top of SDRAM is a second heap, the System Heap,
which is also used by the monitor processor. Neither of these areas
should be accessed by applications.

\rule{\linewidth}{1pt}

\subsection{\itshape Change log:}

\begin{itemize}
\item {\itshape 1.20 - 07aug13 - ST} - initial release - comments to
  {\itshape \Email}
\item {\itshape 1.21 - 05sep13- ST} - minor update for 1.21
\item {\itshape 1.30 - 07apr14- ST} - update for 1.30
\item {\itshape 1.33 - 19sep14- ST} - update for 1.33 - added app\_id
parameter to rtr\_mc\_load.
\item {\itshape 2.0.0 - 08mar16- ST} - update for 2.0.0

\end{itemize}

\end{document}
